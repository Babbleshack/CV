%%%%%%%%%%%%%%%%%%%%%%%%%%%%%%%%%%%%%%%%%
% Medium Length Professional CV
% LaTeX Template
% Version 2.0 (8/5/13)
%
% This template has been downloaded from:
% http://www.LaTeXTemplates.com
%
% Original author:
% Rishi Shah 
%
% Important note:
% This template requires the resume.cls file to be in the same directory as the
% .tex file. The resume.cls file provides the resume style used for structuring the
% document.
%
%%%%%%%%%%%%%%%%%%%%%%%%%%%%%%%%%%%%%%%%%

%----------------------------------------------------------------------------------------
%	PACKAGES AND OTHER DOCUMENT CONFIGURATIONS
%----------------------------------------------------------------------------------------

\documentclass{resume} % Use the custom resume.cls style

\usepackage[left=0.75in,top=0.6in,right=0.75in,bottom=0.6in]{geometry} % Document margins
\usepackage{enumitem}
\newcommand{\tab}[1]{\hspace{.2667\textwidth}\rlap{#1}}
\newcommand{\itab}[1]{\hspace{0em}\rlap{#1}}

\newenvironment{myindentpar}[1]%
{\begin{list}{}%
		{\setlength{\leftmargin}{#1}}%
		\item[]%
	\end{list}}

\name{Dominic Lindsay} % Your name
%\address{Longsight, Manchester, United Kingdom} % Your address
\address{Github: https://github.com/babbleshack}
%\address{Google Scholar: https://scholar.google.co.uk/citations?hl=en&user=97GaVyYAAAAJ}

%\address{123 Pleasant Lane \\ City, State 12345} % Your secondary addess (optional)
\address{ dcrl94@gmail.com \\ d.lindsay4@lancaster.ac.uk} % Your phone number and email


\begin{document}
	
\begin{rSection}{Biography}
	Experienced and passionate Research Engineer with strong demonstrated history of working both academic and software industries Currently working towards a PhD of Computer Science. Research include development of novel resource management and scheduling policies for management of decentralised clusters. Strong experience of managing and deploying large scale distributed systems and infrastructure as well as Operating Systems development for both Linux and embedded systems.  
	%Experienced and passionate Software Engineer with a demonstrated history of working in the computer software industry. Currently working towards a Doctorate of Computer Science at Lancaster University. Interested in Software Engineering and Architecture, Design Patterns, Algorithms and Data Structures,  Distributed Systems, Orchestration, Cloud Computing, Fog Computing and the Internet of Things.
\end{rSection}	

%\begin{rSection} {Background}
%	Software Engineering Specialization with experience in Operating Systems, Embedded Systems, Internet of Things, Cloud and Fog Computing, Software Architecture and Design Patterns.
%\end{rSection}

%----------------------------------------------------------------------------------------
%	EDUCATION SECTION
%----------------------------------------------------------------------------------------

\begin{rSection}{Education}

{\bf Lancaster University, Lancaster, United Kingdom} \hfill {\em September 2017 - Present} 
\\ PhD Computer Science  \hfill {\em Expected Completion: March 2021}

{\bf Lancaster University, Lancaster, United Kingdom} \hfill {\em October 2012 - June 2016} 
\\ Masters in Science (MSci), Software Engineering \hfill {\em 1\textsuperscript{st} Class }

{\bf Loreto College,  Manchester, United Kingdom} \hfill {\em September 2010 - June 2012}
\\ BTEC National Extended Diploma IT Practitioner \hfill {\em A*A*A*}
\end{rSection}

%--------------------------------------------------------------------------------
%    Projects And Seminars
%-----------------------------------------------------------------------------------------------
\begin{rSection}{Research Projects}
{\bf Epimetheus: Intercluster Aware Orchestration}\\
{\textit{PhD Thesis Project}}\\
Large scale distributed applications are supported by  ``clusters'' of shared computing resources. Schedulers are responsible for orchestration of application workloads whilst maintaining cluster invariants. Shared resource clusters are composed of several clusters of resources managed as a single federated clusters, providing infrastructure for any class of application workload. However current federated cluster architecture fail to account for performance implication related to inter-cluster operations. As such they are exposed to volatile network behaviours, dynamic cluster utilisation and heterogenous subcluster scheduling policies. This project aims to develop novel scheduling policies capable of capturing inter-cluster metrics and meet the demands of dynamic workloads executing across decentralised infrastructures.

{\bf Feature Monkeys, a composable framework for Internet of Things applications.}\\
{\textit {MSci Project -- with Demopad Software}}\\
Feature composition framework for development of bespoke IoT Smart-Home applications from \textit{``Software Produce Lines''}. In this project I developed a framework which enables composition of IoT \textit{features} (sensing, actuation), enabling end user to compose application for their smart home environment. The project achieved two goals, development of a \textit{feature} plugin framework for interfacing with new IoT devices, and second, development of a feature composition framework enabling end user to compose  new smart home applications.
\end{rSection}

%--------------------------------------------------------------------------------
%    Publications
%--------------------------------------------------------------------------------
\begin{rSection}{Publications}
		\begin{enumerate}
			\item \textbf{\underline{D.Lindsay}}, S.S.Gill, P.Garraghan --- PRISM: An Experiment Framework for Straggler Analytics in Containerized Clusters (\textbf{\textit{\mbox{Published}}})
			\newline \textit{Proceedings of the 5th International Workshop on Container Technologies and Container Clouds} 
			
				\item \textbf{S.S.Gill}, \textbf{\underline{D.Lindsay}}, P.Garraghan, et al - Transformative effects of IoT, Blockchain and Artificial Intelligence on cloud computing: Evolution, vision, trends and open challenges 
					\newline \textit{Internet of Things}  (\textit{\textbf{in-review}})
				\item \textbf{\underline{D.Lindsay}}, Y.Elkhatib, P.Garraghan - Intercluster aware orchestration for decentralised infrastructure (\mbox{\textbf{\textit{on-going}}})
		\end{enumerate}

\end{rSection}
%----------------------------------------------------------------------------------------
%	TECHNICAL STRENGTHS SECTION
%----------------------------------------------------------------------------------------

\begin{rSection}{Technical Strengths}

\begin{tabular}{ @{} >{\bfseries}l @{\hspace{6ex}} l }
Systems Development  & C, C++, Rust, Java, Python, GoLang\\
Operating Systems & Linux, Windows, BSD, Raspbian (ARM)  \\
Infrastructure Management & Kubernetes, Yarn Ceph, HDFS, Prometheus \\
Deployment  & Docker, LXC,  Ansible, Puppet\\
%Version Control & Git, SVN, IBM Rational/Synergy/Team Concert
\end{tabular}

\end{rSection}
%----------------------------------------------------------------------------------------
%	WORK EXPERIENCE SECTION
%----------------------------------------------------------------------------------------

\begin{rSection}{Industrial Experience}
	

	\begin{rSubsection} {ARM, Cambridge}{June 2020 --- September 2020} {Research Engineer} {Rust, C++, Nix}
	\item[] Part of the security research team investigating Trusted Execution Environments and their applications in distributed edge computation. This project spans formally verified kernel development to implementation of high level application frameworks. 	
	Specifically my work here includes development of dynamic capability and memory management libraries. Whilst a secondary objective during my time at Arm included the design of distributed trusted computing platform used for execution of applications over dataset's owned by mutually distrusting peers across a distributed cluster of TEE enabled hosts's.
	\end{rSubsection}
	
	\begin{rSubsection}{Lancaster University}{September 2017 - Present}{Teaching Assistant}{}
		\item [] Teaching and coursework development for multiple modules including:
			\setlist{nolistsep}
			\begin{itemize}[noitemsep]
			\item Operating Systems -- Concurrency, Memory Management, Filesystems and Linux Kernel Development.
			\item Networking --  Protocols (ICMP, TCP, UDP), and network programming. 
			\item Distributed Systems -- Distributed data processing frameworks (Yarn, Spark), Virtulisation and Isolation, RPC with GRPC and Java RMI.
			%\item Software Development -- Focused on teaching fundamentals of software development such as abstraction, algorithms and data structure implementation.
			%\item Fundamental of Computer Science -- Algorithm and data structure theory, complexity, logic, and set theory.
		\end{itemize}
	\end{rSubsection}

	\begin{rSubsection}{Demopad Software, Lancaster}{September 2017 - January 2019} {Platform Software Engineer}{C++, NodeJS, Bash, Docker}
		\item [] I was responsible for identifying the cause of and fixing bugs in the system platform. Implementation of functional features and devices required for extention of demopads automation platform. Development of an automated build platform, capable of pulling latest changes from a \textit{version control system} and building a release and development image.
		%		\item [] I worked with Demopad part-time. My role here includes fixing bugs in the system platform and implementing new functionality. I mainly work with C++ and NodeJS. I am also responsible for the system build process (automated via jenkins).
	\end{rSubsection}
	
	\begin{rSubsection}{SCISYS, Bristol} {July 2016 - September 2017}{Software Engineer}{Java, C++, JavaScript, NodeJS, Apache ESB}
		\item [] At Scisys I worked on a variety of projects as a software engineer within the company and collaborated with industry partners. These included development of automated validation and testing of safety critical systems including RNLI's SIMS situational awareness and control platform. Worked as part of R\&D developing personal tracking and alert system for outdoor security operation in outdoor environments. Taking a leading role in the design and development of the IoT platform making use of several micro-service frameworks including apache storm and camel.
		%\begin{enumerate}
		%	\item[] \textbf{\textit{IoT Tracking and Control Platform}} Design and early development of an IoT tracking and control system. We developed a  Service Oriented Architecture  using Apache ESB components.  Apache Felix/Ares was chosen as the component framework. Apache Camel was used to provides a routing mechanism for interaction of components at runtime. Finally, I implemented the OGC SensorThings REST API as our standard model for defining IoT devices and their environments.
		%	\item[] \textbf {\textit{RNLI SIMS System}} Development of hardware emulation platform for testing RNLI Situational Awareness and Response Systems (SIMS).  My responsibilities included development of hardware emulators including, CANBUS (GPS, Motor control and AIS ), IP Cameras,  erroneous value injection, and power management. Test case development,used for identifying software errors or bugs, feature regression and performance analysis. I also developed and evolved the SIMS user interface including development and integration of additional widgets reflecting system state.
		%\end{enumerate}
	\end{rSubsection}
\end{rSection}

\begin{rSection}{References}
%	Reference are available on request.
	\begin{description}
		\item[Dr Peter Garraghan, Lancaster University]
		\textit{Lecturer (\textit{Supervisor}}) -- p.garraghan@lancaster.ac.uk
		\item[Dr Yehia Elkhatib, Lancaster University]
		\textit{Senior Lecturer} -- y.elkhatib@lancaster.ac.uk
		\item[Dr Gerald Kotonya, Lancaster University]
		\textit{Senior Lecturer} -- g.kotonya@lancaster.ac.uk
	\end{description}
\end{rSection}	
\end{document}
